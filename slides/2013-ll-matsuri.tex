% -*- coding: utf-8 -*-

\documentclass[slide,papersize]{jsarticle}
\renewcommand\kanjifamilydefault{\mcdefault}
\renewcommand\familydefault{\rmdefault}


\usepackage[dvipdfmx]{graphicx,color,hyperref}
\usepackage{ascmac}
\usepackage{times}

\newcommand{\link}[1]{\textcolor[named]{Blue}{\underline{#1}}}


\begin{document}

\title{Lua組み込みプログラミング}
\author{久井 亨\thanks{Kronecker's Delta Studio}}


\begin{flushright}
 {\small 2013/8/24 LLまつり Lightning Talk}
\end{flushright}

\vspace*{0.8cm}
\begin{center}
 {\Large Lua組み込みプログラミング}

\vspace*{0.3cm}

Kronecker's Delta Studio

\vspace*{0.1cm}

{\large 久井 亨}

\end{center}


\section*{Lua 20周年!}


{\scriptsize
\textbf{Subject}: 20 years of Lua\footnote{http://lua-users.org/lists/lua-l/2013-07/msg00846.html}\\
\textbf{From}: Luiz Henrique de Figueiredo \texttt{<lhf@...>}\\
\textbf{Date}: Sun, 28 Jul 2013 20:07:52 -0300\\
}


\begin{center}
The earliest implementation of Lua that we have found to date has files
dated 28 Jul 1993.

\vspace*{0.2cm}

\link{http://www.lua.org/ftp/lua-1.0.tar.gz}
\end{center}


\section*{デモ内容}

\begin{itemize}
 \item Xcodeを使ってiOS appを作る
 \item メインプログラムをLuaで書く
\end{itemize}


\section*{材料}

\begin{itemize}
 \item Lua
 \item Box2D(物理エンジン)
 \item iOS SDK
       \begin{itemize}
        \item UIKit Framework(PNG画像表示)
        \item Core Graphics Framework(ベクタ描画)
        \item Core Motion Framework(加速度センサ)
       \end{itemize}
\end{itemize}


\section*{糊}

\begin{itemize}
 \item SWIG
       \begin{itemize}
        \item C/C++のコードをLuaから呼び出す
        \item (地雷原)
       \end{itemize}

 \item Lua-Objective-C Bridge
       \begin{itemize}
        \item Objective-CのオブジェクトにLuaからメッセージを送る
        \item (自作)\footnote{\link{https://github.com/torus/Lua-Objective-C-Bridge}}
       \end{itemize}
\end{itemize}

\section*{その他}

\begin{itemize}
  \item Cの構造体を返すObjctive-C APIには泣きながら個別に対応
\end{itemize}

\section*{Fork on GitHub!}


\vspace*{1cm}

\begin{center}
\link{https://github.com/torus/ios-lua-lander}
\end{center}

\end{document}
